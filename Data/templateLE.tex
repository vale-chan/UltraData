\documentclass[11pt]{article}
\usepackage[german]{babel}
\usepackage[utf8]{inputenc}
\usepackage{booktabs}
\usepackage{fancyhdr}
\usepackage{fontspec}
\usepackage[T1]{fontenc}
\usepackage{graphicx}
\usepackage{lastpage}
\usepackage[onehalfspacing]{setspace}
\usepackage{tabularx}
	\newcolumntype{L}[1]{>{\raggedright\arraybackslash}p{#1}}
	\newcolumntype{C}[1]{>{\centering\arraybackslash}p{#1}}
	\newcolumntype{R}[1]{>{\raggedleft\arraybackslash}p{#1}}
\usepackage{tikz}
\usetikzlibrary{calc}
\usepackage{verbatim}
\usepackage{xcolor}
\usepackage[headheight=2.25cm, left=3cm,right=2cm,top=3.5cm,bottom=2cm]{geometry}
\setmainfont{Arial}




   
\begin{document}

\begin{titlepage}

\begin{tikzpicture}[overlay, remember picture]
\node[anchor=north west, %anchor is upper left corner of the graphic
      xshift=-14.61cm, %shifting around
      yshift=-10.8cm] 
     at (current page.north west) %left upper corner of the page
     {\includegraphics[width=42cm]{Kreis}}; 
\node[anchor=north west, %anchor is upper left corner of the graphic
      xshift=15.98cm, %shifting around
      yshift=-25.07cm] 
     at (current page.north west) %left upper corner of the page
     {\includegraphics[scale=.35]{PHSGlogow}}; 
\end{tikzpicture}

\begin{tikzpicture}[overlay, remember picture]
\node[anchor=north west,
      xshift=3cm,
      yshift=-15cm,
      align=left,
      text width=14cm] 
     at (current page.north west)
     {{\color{white} \fontsize{35}{11}\selectfont\baselineskip=40pt Ergebnisse aus der\newline Lehrevaluation\par
     \vspace{1cm}
     \color{white} \fontsize{18}{11}\selectfont Evaluationsbericht __aktuelleStudienjahr__\par
     \vspace{.5cm}
     \color{white} \fontsize{12.5}{11}\selectfont HE\&QM\par}};
\end{tikzpicture}


\end{titlepage}


\pagestyle{fancy}
	\fancyhf{}
	\renewcommand{\headrulewidth}{0pt}
	\rhead{\includegraphics[scale=.35]{PHSGlogo}}
	\fancyfoot[L]{\fontsize{9}{11} \selectfont HE\&QM}
	\fancyfoot[C]{\fontsize{9}{11} \selectfont __monat__ __jahr__}
	\fancyfoot[R]{\fontsize{9}{11} \selectfont Seite \textbf{\thepage}\ von \textbf{\pageref{LastPage}}}



\section{Übersicht durchgeführter Evaluationen}
\label{sec: übersicht}
Im Studienjahr __aktuelleStudienjahr__ wurden an der PHSG insgesamt __totalEvaluations__ durchgeführt; das sind __relativeChangeTotalEvalution__ \% __moreOrLess__ __letztesStudienjahr__.
Der untenstehenden Abbildung kann entnommen werden, wie viele elektronische Lehrevaluationen in den letzten fünf Studienjahre an der Pädagogischen Hochschule St. Gallen insgesamt durchgeführt wurden.

\includegraphics[width=1\textwidth]{../histogramms/__uebersichtAnzahlEvaluationen__}

\newpage
\section{Fixierte Items}
\label{sec: fixierteItems}
Im Studienjahr __aktuelleStudienjahr__ hat das Rektorat __anzahlFixierteItems__ für die elektronische Lehrevaluation fixiert. Für jedes dieser Items ist im folgenden einzeln der Mittelwert und die Standardabweichung über die letzten fünf Studienjahre abgebildet.

\textbf{TIMELINE INNEHAUE}
\bigskip

\includegraphics[width=1\textwidth]{../bars/__item1__}

\includegraphics[width=1\textwidth]{../bars/__item2__}

\includegraphics[width=1\textwidth]{../bars/__item3__}

\includegraphics[width=1\textwidth]{../bars/__item4__}

\includegraphics[width=1\textwidth]{../bars/__item5__}

\includegraphics[width=1\textwidth]{../bars/__item6__}

\includegraphics[width=1\textwidth]{../bars/__item7__}

\includegraphics[width=1\textwidth]{../bars/__item8__}

\includegraphics[width=1\textwidth]{../bars/__item9__}




\newpage
\section{Übersicht des gesamten Fragepools}
\label{sec: fragepool}
Der Fragepool der PHSG umfasst derzeit 106 Items, welche sich in 14 Themenfeldern gliedern. In den folgenden zwei Abbildungen werden sämtliche Items, welche im Studienjahr HS21/FS22 abgefragt wurden, abgebildet. Die Abbildungen geben die Mittelwerte der Themenfelder wieder (Position der dicke, waagrechte Linien), die Mittelwerte der abgefragten Items (Position der Punkte) und die Anzahl Antworten pro Item (Grösse der Punkte). Auf der y-Achse der Abbildungen kann jeweils der Wert des Mittelwertes abgelesen werden. 13 der 14 Themenfelder verwenden eine Skala von 1 bis 4, wobei 1 jeweils die "`schlechteste"' Antwortoption darstellt (z. B. \textit{stimme gar nicht zu}, \textit{trifft gar nicht zu}, etc.) und 4 die "`beste"' Antwortoption (z. B. \textit{stimme sehr zu}, \textit{trifft sehr zu}, etc.). Eines der Themenfelder, das Themenfeld \textit{Anspruchsniveau} verwendet eine Skala von 1 bis 5, wobei sowohl 1 als auch 5 eine "`schlechte"' Antwortoption darstellen und der Mittelpunkt 3 die "`beste"' Antwortoption repräsentiert (z. B. \textit{1 = zu wenig}, \textit{2 = eher zu wenig}\textit{3 = genau richtig}, \textit{4 = eher zu viel}, \textit{5 = zu viel}). \\

\noindent
Insgesamt fallen XXX Items unter einen Mittelwert von 3. Dabei handelt es sich um die folgenden Items:
\begin{itemize}
      \item Themenfeld \textit{XXX}: \textit{Wording Item} (\textit{n} = XXX; \textit{m} = XXX)
\end{itemize}

\includegraphics[angle = 90, height=\textheight]{../lollipop/lollipop4.png}

\includegraphics[angle = 90, scale=.58]{../lollipop/lollipop5.png}
\end{document}

